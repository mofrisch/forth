\section{Testing}

\subsection{Bit Fields}

The bit fields words were described in the previous chapter, when I finished the
implementation of them, I wrote a simple shell script (listing \ref{sbitfiels})
which made sure that I discovered all my typos:

\begin{lstlisting}[float, caption=Bit fields test script, label=sbitfiels]
#!/bin/sh
# script: inst_field_test.sh
# purpose: test disasm-<number> and 
# disasm-<number>,<number> words

gforth='~/gforth-20050128-ppc64/bin/gforth-fast'
disasm='~/praktikum/ppc/disasm.fs'

# ranges
echo "testing single:";
echo "---------------";
for k in $(cat befehle_binaer);
 do
   for l in $(cat to_test);
    do
      first=`echo $l | cut -d, -f1`
      second=`echo $l | cut -d, -f2`
      len=`echo $second - $first + 1 | bc`
      echo -n "$k disasm-$first,$second: ";
      $gforth $disasm -e "%$k disasm-$first,$second 
        dup hex.  %${k:$first:$len} dup hex. = . bye"
      echo ""
    done
done

# single bits
echo "testing single bits:";
echo "--------------------";
for k in $(cat befehle_binaer);
do
    for l in $(cat to_test2);
    do
        bit=`echo $l`
        echo -n "$k disasm-$bit: ";
        $gforth $disasm -e "%$k disasm-$bit 
            dup hex. %${k:$bit:1} dup %hex. = . bye"
        echo ""
    done
done
\end{lstlisting}

First I defined two global variables which are used to load the
\texttt{disasm.fs} source file and to identify the path to the 
the Gforth binary. Two files \texttt{to\_test} and \texttt{to\_test2} are used
to identify the number(s) which are to be inserted behind \texttt{disasm-}.
In a file called \texttt{befehle\_binaer}, I created some random strings which
have the length 32, because an instruction is 32 bit long (befehle is german for
instructions and binaer for binary) and they represent a single instruction. 
Since those strings are binary they are either composed of \texttt{0} or 
\texttt{1}. Those files may be located in any directory but as this script is
implemented currently, every file has to be in the same directory. This script
works only if every \texttt{disasm-<n1>[,<n2>]} words are defined in gforth's
standard word list.

In the first block, words of the form \texttt{disasm-<n1>,<n2>} are
tested. For every in line the \texttt{befehle\_binaer} file every possible 
\texttt{disasm-<n1>,<n2>} word is tested. In the second nested loop the most 
interesting lines are:

\begin{verbatim}
echo -n "$k disasm-$first,$second: ";
$gforth $disasm -e "%$k disasm-$first,$second 
    dup hex. %${k:$first:$len} dup hex. = . bye"
\end{verbatim}

The first line displays the instruction encoded in binary and the
\texttt{disasm} word which is applied. In the second line the output
of \texttt{disasm-<n1>,<n2>} is compared with a reference. The comparison is
done with the \texttt{=} Gforth word. If the result is same with the reference
true should be displayed which is \texttt{-1} in Gforth. For debugging purposes
the reference and the result is displayed. The \texttt{\$\{k:\$first:\$len\}}
shell command extracts the reference from \$k which represents the whole
instruction. A string which starts at \$first and is \$len long is extracted.
The block which is used to test words of type \texttt{disasm-<n>} is quite the
same with the exception that it used the \texttt{to\_test2} file. It was easy to
identify failed test cases, every line which had a \texttt{0} at the end, was a
\texttt{FAILED} test case.

When we call this script it produces output like listed in listing \ref{sbitout}
(actually it is much more output than this).

\begin{lstlisting}[float, language=csh, caption=Bit field test output,
label=sbitout]
testing single:
---------------
10001110111100110100010110100100 disasm-0,5: $23 $23 -1 
10100001101000101011001111000011 disasm-0,5: $28 $28 -1 
11010001101000101011001111000011 disasm-0,5: $34 $34 -1 
11110101010000111110110000101010 disasm-0,5: $3D $3D -1
.
.
.

testing single bits:
--------------------
10001110111100110100010110100100 disasm-31: $0 $0 -1 
11110001001000110100010101101011 disasm-31: $1 $1 -1 
.
.
.
\end{lstlisting}

\subsection{Test Structure}

In order to test the assembler/disassembler I had to create a testing structure
where I had all my test cases. I created this structure:

\begin{verbatim}
micrev@north:~/praktikum/test/mnemonic$ ls
a   find_mnemonic.sh  make_unique.py test_all_forms.sh    xfl  xs     
b   find_mnemonics.sh md             test_asm.py          xfx 
d   i                 mds            test_disasm-inst.py  xl
ds  m                 sc             x                    xo
\end{verbatim}

In this listing, all names which are the same as the forms of the
\texttt{ppc-\{32,64\}} architecture are directories.
Those directories contain mainly text files with instructions, those
were generated using \texttt{objdump} and 3 scripts.

The script in listing \ref{smnem} was used in order to search through all
binaries available at a machine for a particular mnemonic of a particular form:

\begin{lstlisting}[float, caption=Script: find\_mnemonic.sh, label=smnem]
#!/bin/sh

# Autor Michal Revucky 
# Purpose: find a certain mnemonic specified by $1 and 
#          write to a file  it writes only a specified 
#          number of lines into to result file which 
#          is determined by $COUNT, the file name 
#          contains the mnemonic too. it checks all 
#          binaries from DIRS, it places the result file 
#          into ./<form>/$HOSTNAME.$1
#          Usage: ./find_mnemonic <mnemonic> <form>

DIRS="/home/complang/micrev/gforth-20050128-ppc64/bin 
/bin /usr/bin /usr/local/bin /usr/X11R6/bin 
/usr/powerpc64-unknown-linux-gnu/gcc-bin/3.4.3 
/opt/Ice-2.0.0/bin"

COUNT=100

if [ $# -ne 2 ]; then 
    echo "usage: $0 <mnemonic> <form>"
    exit
fi

if [ ! -d $2 ]; then 
    mkdir $2
fi

for k in $DIRS; do
 for l in `ls $k`; do
  objdump -d $k/$l | grep $1 
    | head -n1 >> $2/$HOSTNAME.$1
   if [ `wc -l $2/$HOSTNAME.$1 
      | grep -o [0-9]*` -ge $COUNT 
      ] ; 
   then exit
   fi
  done
done
\end{lstlisting}

For each Form I created a file called \texttt{mnemonics} in other words: it was 
in every directory as mentioned above and contained every mnemonic for a 
particular form. So I wrote another script \texttt{find\_mnemonics.sh} 
(listing \ref{smnems}) which would be faster, this time you had only to 
specify the form and it did exactly the same as the script 
\texttt{find\_mnemonic.sh}, but for every mnemonic of the specified form.

\begin{lstlisting}[float, caption=Script: find\_mnemonics.sh, label=smnems]
#!/bin/sh

# Autor Michal Revucky
# Purpose: find certain mnemonics of form specified 
#          by $1 and write to a file it writes only
#          a specified number of lines into to result
#          file which is determined by $COUNT, the 
#          file name contains the mnemonic too. it 
#          checks all binaries from DIRS, it places 
#          the result file into ./<form>/$HOSTNAME.$1
#          it takes every mnemonic from $1/mnemonic
# Usage: ./find_mnemonic <form>

DIRS="/home/complang/micrev/gforth-20050128-ppc64/bin 
/bin /usr/bin /usr/local/bin /opt/Ice-2.0.0/bin
/usr/powerpc64-unknown-linux-gnu/gcc-bin/3.4.3 
/usr/X11R6/bin"

COUNT=100

if [ $# -ne 1 ]; then 
    echo "usage: $0 <form>"
    exit
fi

if [ ! -d $1 ]; then 
    mkdir $1
fi

for j in `cat $1/mnemonics`; do
 echo "checking $j"
 touch $1/$HOSTNAME.$j;
  for k in $DIRS; do
    for l in `ls $k`; do
      if [ `wc -l $1/$HOSTNAME.$j 
         | grep -o [0-9]*` -lt $COUNT 
         ] ; then 
         objdump -d $k/$l | grep $j 
            | head -n1 >> $1/$HOSTNAME.$j ;
      fi
    done
  done
done
\end{lstlisting}

The files which contain the test cases have a form as shown in listing 
\ref{stfile}.

\begin{lstlisting}[float, caption=Example of a test file, label=stfile]
    100117b0:   7d 8c 12 14     add     r12,r12,r2
    1001132c:   7c 1c ba 14     add     r0,r28,r23
    100112e8:   7c 00 52 14     add     r0,r0,r10
    .
    .
    .
\end{lstlisting}

After the execution of \texttt{find\_mnemonics.sh} I checked the files which
contained the instructions and what I found was that their content was not
unique i.e. \texttt{add r0,r0,r1} appeared in the file more the once. From the
point of view of testing it would not matter, but I wanted the files unique, so
I wrote a Python (listing \ref{suni}) script:

\begin{lstlisting}[float, language=python, caption=Script: make\_unique.py, 
label=suni]
#!/usr/bin/env python

# makes testfiles, form: <hostname>.<mnemonic> unique
# $1 specifies the form to make unique

import commands
from optparse import OptionParser

HOSTNAME = commands.getoutput("hostname")
MNEMONIC_TEST='~/praktikum/test/mnemonic'

def get_opc_code_hex(line) :
    l = line.split()
    return l[1]+l[2]+l[3]+l[4]

    if __name__ == '__main__' :
        parser = OptionParser("%prog <form>")
        (options, args) = parser.parse_args()
        if len(args) < 1 or len(args) > 1 :
            parser.error("incorrect number of arguments")
        dir = MNEMONIC_TEST + '/' + args[0]
        for k in commands.getoutput(
                'ls %s/%s*'
                %(dir, HOSTNAME)).split('\n') :
            print k
            f = open(k,'r')
            lines = f.readlines()
            new_file = open(k+'.unique', 'w') 
            tmp_list = []
            new_file_list = []
            for l in lines :
                opc_code_hex = get_opc_code_hex(l)
                if opc_code_hex not in tmp_list :
                    tmp_list.append(opc_code_hex)
                    new_file_list.append(l)
            new_file.writelines(new_file_list)
            new_file.close()
            print commands.getoutput('mv 
                %s.unique %s' %(k,k))
\end{lstlisting}

For all test files of the form \texttt{<hostname>.<mnenonic>} this script 
iterates over every single line of any given test file of a particular form. 
For each line the machine code in the hexadecimal form is parsed from the file,
a list holds all hexadecimal numbers which are already used,
only instructions which are not already in this list are written to the new
test file which is called \\\texttt{<hostname>.<mnemonic>.unique}, finally this
file is renamed to its final name.

\subsection{Mnemonic Tests}

When I set up the directories with the test cases I had to write some scripts
which would parse the instructions from the test files, call \texttt{gforth} and
compare the output with a reference. For the disassembler I wrote the script
called \texttt{test\_disasm-inst.py} and \texttt{test\_asm.py} for the 
assembler . I decided to use Python for this, because it has a lot of useful 
libraries i.e. regular expression for parsing.

\subsubsection{Disassembler}

The usage of the \texttt{test\_disasm-inst.py} is:

\begin{verbatim}
$ ./test_disasm-inst.py -h
usage: test_disasm-inst.py [-m] [-a] form [mnemonic]

options:
-h, --help      show this help message and exit
-m, --mnemonic  compare the returned mnemonic
-a, --args      compare the result args
\end{verbatim}

As you can see, you have to specify a form to be tested, if you want to test a
particular mnemonic only, you can specify an optional mnemonic. The options are
not really useful, because I always called the script with \texttt{-m} and
\texttt{-a}. When the script was called, then for each test case in a directory
it iterated over each line of a test file and did this:

\begin{enumerate}

    \item parsed a line which had the form as mentioned in listing \ref{stfile}
    and created a Python tuple which is easier to use in a Python script.
    \item call Gforth with arguments which ware taken from the tuple
    created in the previous step.
    \item compare the results returned by Gforth and display a status either
    "OK" or "FAILED".

\end{enumerate}

The output in listing \ref{sdout} is produced by the 
\texttt{test\_disasm-inst.py} script, the first tuple (followed by "testing") 
is the information which is parsed from the test 
files. The first element is the address of the instruction, followed by the 
instruction in hex, then by the mnemonic and finally by the arguments of the 
instruction wrapped in a python list. Either "OK" or "FAILED" denotes the status
of this particular instruction which is tested. The tuple which is followed by
"disasm-inst" contains elements which is returned by the \texttt{disasm-inst}
word. The first element is the returned mnemonic followed by the arguments which
were disassembled wrapped in a Python list. These two elements are parsed from
the third one, which is Gforth's output. As you can see the second and third
element of the first tuple is compared with the first and second element of the
second tuple, if they match this part of a test case succeeds. When you run this
with \texttt{-m} and \texttt{-a} option it also displays a number of failed
test cases.

\begin{lstlisting}[float, caption=Output of test-disasm-inst.py, label=sdout]
Testing mnemonic and its arguments:
whole form from ~/praktikum/test/mnemonic/md
=========================================================
testing: ( '100022d0' , '782106e9'
         , 'rldic.'
         , ['1', '1', '0', '59']
         ) OK
disasm-inst: ( 'rldic.'
             , ['1', '1', '0', '59']
             , '1 1 0 59 rldic.'
             )
---------------------------------------------------------
testing: ( '100012c0'
         , '782106e8'
         , 'rldic'
         , ['1', '1', '0', '59']
         ) OK
disasm-inst: ( 'rldic'
             , ['1', '1', '0', '59']
             , '1 1 0 59 rldic'
             )
---------------------------------------------------------
\end{lstlisting}

\subsubsection{Assembler}

The script \texttt{test\_asm.py} is quite the same as the script
\texttt{test\_disasm-inst.py}. Since I wrote this script after the disassembler
script I knew that I would not need a bunch of options so the usage is quite
simple:

\begin{verbatim}
$ ./test_asm.py -h
usage: test_asm.py <form>

options:
  -h, --help  show this help message and exit
\end{verbatim}

This time the script works the other way:

\begin{enumerate}
    \item parse the processed line and create the tuple followed by "testing",
    as you may see in the output of \texttt{test\_asm-inst.py} script
    (listing \ref{saout}), the elements are the same as in listing \ref{sdout}.
    \item call Gforth, the first element of the second tuple from listing
    \ref{saout}, is the command Gforth is called with. The second element in 
    the second tuple is the result and it is compared with the second element 
    of the first tuple.
    \item if the result matches the reference, "OK" is displayed, the script
    continues.
\end{enumerate}

In order to use the script the way it is currently implemented you should
comment the words which make the assembling words appear in the word list
\texttt{assembler} and uncomment the additional output in the \texttt{h,} word
which causes to output the machine code of each instruction which is being
assembled.

\begin{lstlisting}[float, caption=Output of test\_asm.py, label=saout]
Testing mnemonic and its arguments:
whole form from ~/praktikum/test/mnemonic/md
=========================================================
testing: ( '100022d0', '782106e9'
         , 'rldic.'
         , ['1', '1', '0', '59']
         ) OK
asm-inst: (' 1 1 0 59 rldic.', '$782106E9')
---------------------------------------------------------
testing: ( '100012c0'
         , '782106e8'
         , 'rldic'
         , ['1', '1', '0', '59']
         ) OK
asm-inst: (' 1 1 0 59 rldic', '$782106E8')
---------------------------------------------------------
\end{lstlisting}

I also created a simple wrapper shell script (listing \ref{sallforms}) which 
calls the two mentioned python scripts and displays only "interesting" stuff.

\begin{lstlisting}[caption=Script: test\_all\_forms.sh, label=sallforms]
#!/bin/sh

FORMS="a b d ds i m md mds sc x xfl xfx xl xo xs"

echo "disassembler"
for k in  $FORMS; do
    ./test_disasm-inst.py -m -a $k 
        | egrep 'form|Testcases' ;
    echo "==============="
done

echo "assembler"
for k in  $FORMS; do
    ./test_asm.py $k | egrep 'form|Testcases' ;
    echo "==============="
done
\end{lstlisting}
